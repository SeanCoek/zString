\documentclass{llncs}

\usepackage{amsmath}
\usepackage{amssymb}
\usepackage{comment}
\usepackage{url}
\usepackage{listings, tikz}
\usetikzlibrary{positioning,shapes}
\usepackage{verbatimbox}

\newtheorem{Definition}{Definition}
\newtheorem{Theorem}{Theorem}
\newtheorem{Lemma}{Lemma}
\newtheorem{Property}{Property}
\newtheorem{Corollary}{Corollary}
\newtheorem{Proof}{Proof}

\renewcommand{\ttdefault}{pcr}

\lstset{language=Java,
	breaklines=true,
	%basicstyle=\small,
	basicstyle=\ttfamily\footnotesize,
	keywordstyle=\bfseries,
	numbers=left,numberstyle=\scriptsize,
	columns=fullflexible, keepspaces=true,
	%numbers=left,numberstyle=\scriptsize
	%numbers=left,numberstyle=\tiny
	%frame=single
}

\newcommand{\mathword}[1]{{\tt \mathit{#1}}}
\newcommand{\mw}[1]{\mathword{#1}}
\newcommand{\keyword}[1]{\mathsf{#1}}
\newcommand{\kw}[1]{\keyword{#1}}
\newcommand{\ruledef}[3]{ $\frac{\begin{array}[c]{c}{ \rulename{#1}}\\ #2 \end{array}}{\begin{array}[c]{c}#3\end{array}}$}
\newcommand{\ruledefN}[2]{ $\frac{\begin{array}[c]{c} #1 \end{array}}{\begin{array}[c]{c}#2\end{array}}$}
\newcommand{\ruledefX}[2]{ $\begin{array}[c]{c} \rulename{#1}\\ #2 \end{array}$}
\newcommand{\rulename}[1]{{\scriptsize\textsc{[\MakeUppercase{#1}]}}}
\newcommand{\rn}[1]{\rulename{#1}}
\newcommand{\textcode}[1]{\lstinline|#1|}
\newcommand{\tc}[1]{\lstinline|#1|}

\newcommand{\Reducesto}[0]{ \ \Downarrow \ }
\newcommand{\equalson}[1]{ \ \stackrel{#1}{=} \ }
\newcommand{\approxon}[1]{ \ \stackrel{#1}{\approx} \ }
\newcommand{\simon}[1]{ \ \stackrel{#1}{\sim} \ }

\newcommand{\kwnull}[0]{\keyword{null}}
\newcommand{\kwnew}[0]{\keyword{new}}
\newcommand{\kwextends}[0]{\keyword{extends}}
\newcommand{\kwclass}[0]{\keyword{class}}
\newcommand{\kwthis}[0]{\keyword{this}}
\newcommand{\kwif}[0]{\keyword{if}}
\newcommand{\kwthen}[0]{\keyword{then}}
\newcommand{\kwelse}[0]{\keyword{else}}
\newcommand{\kwskip}[0]{\keyword{skip}}

\newcommand\Var{\mathtt{VAR}}
\newcommand\Val{\mathtt{V}}
\newcommand\Obj{\mathtt{OBJ}}

%%% The key relations

\newcommand{\VPT}{\Omega}
\newcommand{\HPT}{\Phi}
\newcommand{\Class}{\mathcal{C}}
\newcommand{\Field}{\mathcal{F}}
\newcommand{\bigO}{\mathcal{O}}
\newcommand{\word}[1]{\langle #1\rangle}
\newcommand{\Nat}{\mathbb{N}}
\newcommand{\less}{\sqsubseteq}
\newcommand{\tflow}{\dashrightarrow}
\newcommand{\hflow}{\rightarrow}
\newcommand{\lhflow}[1]{\stackrel{#1}{\hflow}}

\newcommand\Loc{\mathcal{L}}
\newcommand\Label{\Phi}
\newcommand\set[1]{\{#1\}}
\newcommand\power{\mathcal{P}}
\newcommand\join{\cup}
\newcommand\subtype{\subseteq}

\title{A Relational Static Semantics for Call Graph Resolution}
\author{\today}
\institute{Jinan University}

\begin{document}

\maketitle

\abstract{The problem of resolving virtual method and interface calls in Object-Oriented languages has been a long standing challenge to the program analysis community. In this paper, we propose a new approach called type flow analysis that represent the propagation of type information between program variables by a group of relations without the help of a heap abstraction. We prove that regarding the precision on reachability of class information to variables, our method produces results equivalent to that one can derive from a points-to analysis. Moreover, in practice, our method consumes lower time and space usage, as supported by the experimental results.
}

\section{Experiment}\label{sec:experiment}
\begin{table}[!htbp]\centering
\caption{Callsite with different analysis}
\begin{tabular}{lcccc}
	\hline
	\textbf{bench} & \textbf{CS$_{origin}$} & \textbf{CS$_{cha}$} & \textbf{CS$_{pta}$} & \textbf{CS$_{tfa}$} \\
	\hline
	compress & 153 & 160 & 18 & 73 \\
	crypto & 302 & 307 & 62 & 121 \\
	bootstrap & 657 & 801 & 891 & 328 \\
	commons-codec & 1162 & 1372 & 270 & 442 \\
	junit & 3196 & 17532 & 11176 & 1358 \\
	commons-httpclient & 6817 & 17118 & 567 & 2927 \\
	serializer & 4782 & 9533 & 1248 & 1756 \\
	xerces & 24579 & 56252 & 10631 & 8111 \\
	eclipse & 23607 & 95073 & 70016 & 9379 \\
	derby & 69537 & 180428 & 85212 & 16381 \\
	xalan & 57430 & 155866 & / & 18669 \\
	antlr & 62007 & 147014 & / & 17177 \\
	jython & 129332 & 466167 & / & /  \\
	batik & 56877 & 235071 & / & 20901 \\
	\hline
\end{tabular}
\label{experiment:Callsite}
\end{table}

\begin{table}[!htbp]\centering
\caption{Time cost with different analysis}
\begin{tabular}{lcccc}
	\hline
	\textbf{bench} & \textbf{T$_{cha}(s^{-1})$} & \textbf{T$_{pta}(s^{-1})$} & \textbf{T$_{tfa}(s^{-1})$} & \textbf{Relations} \\
	\hline
	compress & 0.17 & 1.46 & 0.12 & 233\\
	crypto & 0.12 & 1.18 & 0.52 & 295\\
	bootstrap & 240.87 & 324.04 & 0.27 & 545\\
	commons-codec & 0.08 & 0.89 & 0.64 & 2235\\
	junit & 249.57 & 324.27 & 1.55 & 6501\\
	commons-httpclient & 0.08 & 1.19 & 2.59 & 10742 \\
	serializer & 216.58 & 304.19 & 1.62 & 9126\\
	xerces & 224.87 & 316.98 & 19.9 & 63755\\
	eclipse & 207.84 & 391.89 & 16.78 & 39984\\
	derby & 241.29 & 484.73 & 110.77 & 181272\\
	xalan & 785.04 & / & 27.58 & 102889\\
	antlr & 474.66 & / & 28.39 & 90829\\
	jython & 3811.02 & / & /  & / \\
	batik & 503.98 & / & 37.97 & 140648 \\
	\hline
\end{tabular}
\label{experiment:Time Cost}
\end{table}

\begin{table}[!htbp]\centering
\caption{Optimalization result}
\begin{tabular}{lcccc}
	\hline
	\textbf{bench} & \textbf{Node$_{origin}$} & \textbf{Node$_{opt}$} & \textbf{Reduce} & \textbf{Time$(s^{-1})$} \\
	\hline
	compress & 154 & 102 & $33.77\%$ & 0.04 \\
	crypto & 251 & 120 & $52.19\%$ & 0.06 \\
	bootstrap & 442 & 223 & $49.55\%$ & 0.11 \\
	commons-codec & 1358 & 639 & $52.95\%$ & 0.87 \\
	junit & 5566 & 3099 & $44.32\%$ & 19.65 \\
	commons-httpclient & 9003 & 4722 & $47.55\%$ & 44.76 \\
	serializer & 6292 & 3478 & $44.72\%$ & 24.40 \\
	xerces & 37674 & / & / & /\\
	eclipse & 33101 & / & / & /\\
	derby & 104315 & / & / & /\\
	xalan & 74547 & /  & / & /\\
	antlr & 54213 & / & / & /\\
	jython & / & / & / & /\\
	batik & 83834 & / & / & /\\
	\hline
\end{tabular}
\label{experiment:Optimalization}
\end{table}


\bibliographystyle{plain}
\bibliography{literature}
\end{document}
