% This is file FAC2egui.tex
% v2.00, 13th July 2000
% (based on FACguide.tex v1.12)

% Copyright (C) 1999,2000 Cambridge University Press

\NeedsTeXFormat{LaTeX2e}

% The following saves the original definitions of \geq and \leq (guide only).
\let\realgeq\geq
\let\realleq\leq

\documentclass{fac}

\ifprodtf \else \usepackage{latexsym}\fi

% The following macros automatically define symbols to be used in Table 1 of
% the authors' guide, using characters from the AMS symbol font MSAM.

\newcommand\black{\ensuremath{\blacktriangleright}}
\newcommand\white{\ensuremath{\vartriangleright}}

\newif\ifamsfontsloaded
\ifprodtf
  \newcommand\whbl{\white\kern-.1em--\kern-.1em\black}
  \newcommand\blwh{\black\kern-.1em--\kern-.1em\white}
  \newcommand\blbl{\black\kern-.1em--\kern-.1em\black}
  \newcommand\whwh{\white\kern-.1em--\kern-.1em\white}
  \amsfontsloadedtrue
\else
  \checkfont{msam10}
  \iffontfound
    \IfFileExists{amssymb.sty}
      {\usepackage{amssymb}\amsfontsloadedtrue
       \newcommand\whbl{\white\kern-.125em--\kern-.125em\black}%
       \newcommand\blwh{\black\kern-.125em--\kern-.125em\white}%
       \newcommand\blbl{\black\kern-.125em--\kern-.125em\black}%
       \newcommand\whwh{\white\kern-.125em--\kern-.125em\white}}
      {}
  \fi
\fi

%% Macros for the guide only %%
\newcommand\eg{\textit{e.g.\ }}
\newcommand\etc{\textit{etc}}
\newcommand\hatp{\skew3\hat{p}}
\newcommand\lra{\ensuremath{\quad\longrightarrow\quad}}
\providecommand\AMSLaTeX{AMS\,\LaTeX}
%% End of macros for the guide %%

\newtheorem{theorem}{Theorem}[section]

\title[Formal Aspects of Computing: \LaTeXe\ Submissions]
      {Formal Aspects of Computing:\\
       \LaTeXe\ Guide for Authors}

\author[M. Reed and C. Notarmarco]
    {Mark Reed$^1$ and Christiane Notarmarco$^2$\\
     $^1$Electronic Products and Composition Group,\\
     Cambridge University Press, Cambridge,\\
     $^2$Springer-Verlag London Limited, Godalming, Surrey, UK}

\correspond{Christiane Notarmarco, Springer-Verlag London
            Limited, Sweetapple House, Catteshall Road,
            Godalming, Surrey GU7 3DJ, UK.
            e-mail: chris@svl.co.uk}

\pubyear{2000}
\pagerange{\pageref{firstpage}--\pageref{lastpage}}

\begin{document}
\label{firstpage}

\makecorrespond

\maketitle

\begin{abstract}
This guide is for authors who are preparing papers for the
\emph{Formal Aspects of Computing} journal using the \LaTeXe\ %
document-preparation system and the Formal Aspects of Computing
class file (\texttt{fac.cls}).
\end{abstract}

\begin{keywords}
\LaTeXe; Class file; \verb"fac.cls"; User guide
\end{keywords}

\section{Introduction}

In addition to the standard submission of hardcopy from authors,
\emph{Formal Aspects of Computing} now accepts machine-readable forms of papers
in \LaTeXe. The layout design for the \emph{Formal Aspects of Computing} journal
has been implemented as a \LaTeXe\ class file, based on the \verb"article"
class as discussed in the \LaTeX\ manual (2nd edition) \cite{Lam:LaTeX}.

Commands which differ from the standard \LaTeX\ interface, or which are
provided in addition to the standard interface, are explained in this
guide (which is \emph{not} a substitute for the \LaTeX\ manual itself).

Note that the final printed version of papers will use the Monotype Times
typeface rather than the Computer Modern available to authors. For this reason
line and page breaks will change and authors should not insert hard breaks
in their text.

Authors planning to submit their papers in \LaTeXe\ are advised to use
\verb"fac.cls" as early as possible in the creation of their files.

\subsection{Introduction to \LaTeX}

\LaTeX\ is constructed as a series of macros on top of the \TeX\ typesetting
program. \LaTeX\ adds to \TeX\ a collection of facilities which simplify
typesetting for authors by allowing them to concentrate on the logical
structure of the document rather than its visual layout. Careful use of the
\LaTeX\ mark-up philosophy results in uniform layout rather than the
\emph{ad hoc} results of some word-processing systems. Authors are advised to
let the defaults control font selection etc., rather than tinker themselves.

The \LaTeX\ system provides a consistent and comprehensive document preparation
interface. Among other things, \LaTeX\ can automatically number list
entries, equations, figures, tables and footnotes, as well as sections and
subsections. Using this numbering system, bibliographic citations, page
references and cross references to any other numbered entity (e.g. sections,
equations, figures) are straightforward.

\subsection{The FAC document class}

The use of document classes allows a simple change of style (or style option)
to transform the appearance of your document. The Springer-Verlag London Limited
(SVL) FAC class preserves the standard \LaTeX\ interface such that any document
which can be produced using the standard \LaTeX\ \verb"article" class can also be
produced with the FAC class.
However, the measure (or width of text) is different from that
for ARTICLE; therefore line breaks will change and it is possible that
longer equations may need re-setting.
Authors are urged to use \verb"fac.cls" from the beginning of their document
preparation, otherwise longer lines may require re-formatting at a later stage.

\subsection{General style issues}

Use of \LaTeX\ defaults will result in a pleasing uniformity of layout
and font selection. Authors should resist the temptation to make
\emph{ad hoc} changes to these. Also avoid use of direct formatting unless
really necessary. Papers will be edited as usual, and this process may be
obstructed by the use of inserted line breaks, etc.

For general style issues, authors are referred to the `Preparation of
manuscripts' in the back cover of the journal. Authors who are interested in
the details of style are referred to \cite{But:Copy} and \cite{Chicago}. The
language used in the journals is British English, and spelling should conform
to this.

Use should be made of symbolic references (\verb"\ref") in order to
protect against late changes of order, etc.

\subsection{Submission of \LaTeX\ articles}

Authors who intend to submit a \LaTeX\ article to FAC should obtain a
copy of the journal's class file \verb"fac.cls". This is available on
request in the first instance from Springer-Verlag London Limited; remember to
specify the type of media you require. Alternatively, you can download
the files at the journal's home page via
\texttt{http://www.springeronline.com}\footnote{do a ``\textsf{SPRINGER
SEARCH}" for the journal name or a ``\textsf{SPRINGER SEARCH -- BY ISBN/ISSN}"
and click on ``\textsf{Download LaTeX macro package (ZIP archive)}"}

If you cannot obtain the proper journal class file, use the standard
\verb"article" class with the default `\verb"10pt"' option.

When submitting the final article, ensure that  the following are included and
are clearly labelled.
\begin{enumerate}
  \item A hardcopy printout of the article.
  \item The input file (exactly matching the hardcopy).
  \item A copy of all user-defined macros.
  \item If you have used \textsc{Bib}\TeX, the \verb".bib", \verb".bbl"
        and \verb".bst" files that were used.
  \item Any other files necessary to prepare the article for typesetting.
\end{enumerate}
Submit this, if at all possible by e-mail, and the hardcopy by post
to D.\,J. Cooke (see the inside-cover of the journal for correct e-mail
and postal addresses). If you do not have access to e-mail, send the files
on disc -- PC or Macintosh $3\frac{1}{2}$ in. -- along with the hard copy
to D.\,J. Cooke.

\section{Using the FAC class file}

First, copy the file \verb"fac.cls" into an appropriate subdirectory on
your system. The FAC class is implemented as a complete document class,
and \emph{not} as an class option.
In order to use the FAC class, replace \verb"article" by \verb"fac" in the
\verb"\documentclass" command at the beginning of your document: that is,
%
\begin{verbatim}
  \documentclass{article}
\end{verbatim}
%
is replaced by
%
\begin{verbatim}
  \documentclass{fac}
\end{verbatim}
%
Author-defined macros should be inserted before \verb"\begin{document}",
or in a separate file and should be included with the submission.
Authors must not change any of the macro definitions or parameters
in \verb"fac.cls".

\subsection{Document class options}\label{sec:ClassOp}

In general, the following standard document class options should \emph{not} be
used with the FAC class file:
%
\begin{itemize}
  \item \texttt{10pt}, \texttt{11pt} and \texttt{12pt} -- unavailable;
  \item \texttt{twoside} is the default (\texttt{oneside} is disabled);
  \item \texttt{onecolumn} is the default (\texttt{twocolumn} is disabled);
  \item \texttt{titlepage} is not required and is disabled;
  \item \texttt{fleqn} and \texttt{leqno} should not be used
        (\verb"fleqn" is already incorporated into the FAC class).
\end{itemize}
%
\ifprodtf
The following new class options are provided:
\begin{itemize}
  \item \texttt{prodtf} -- tells the class file that we want to use the
    production typeface. This automatically resets the odd, even and top
    margins.
\end{itemize}
\fi

\section{Additional facilities}

In addition to all the standard \LaTeX\ design elements, the FAC class
includes the following features.
%
\begin{enumerate}
  \item Additional commands for typesetting the title page. Extended
        commands for specifying a short version of the title and author(s)
        for the running headlines.
  \item The \verb"\correspond" and \verb"\makecorrespond" commands for printing
        the name and address for further correspondence.
  \item The \verb"keywords" environment.
  \item Extended \verb"\newtheorem" command which can produce
        unnumbered environments.
  \item Proof environment.
  \item Control of enumerated lists.
\end{enumerate}
%
Once you have used these additional facilities in your document,
it can be processed only with \verb"fac.cls".

\subsection{Titles and author's name}

In the FAC style, the title of the article and the author's name (or
authors' names) are used both at the beginning of the article for the main
title and throughout the article as running headlines at the top of every
page.
The title is used on odd-numbered pages (rectos) and the author's name
appears on even-numbered pages (versos).
Although the main heading can run to several lines of text, the running
head line must be a single line.
Moreover, the main heading can also incorporate new line commands
(\eg \verb"\\") but these are not acceptable in a running headline.
To enable you to specify an alternative short title, which should not be
more than 48 characters and spaces, and an alternative short author's name,
the standard \verb"\title" and \verb"\author" commands have been extended
to take an optional argument to be used as the running headline. \eg
%
\begin{verbatim}
  \title[Formal Aspects of Computing: \LaTeXe\ Submissions]
        {Formal Aspects of Computing:\\
         \LaTeXe\ Guide for Authors}

  \author[M. Reed and C. Notarmarco]
      {Mark Reed$^1$ and Christiane Notarmarco$^2$\\
       $^1$Electronic Products and Composition Group,\\
       Cambridge University Press,Cambridge,\\
       $^2$Springer-Verlag London Limited, Godalming, Surrey, UK}
\end{verbatim}

You may wish to add a \verb"\thanks" note, which produces a footnote to the
title or author. In this case, the superscripts by the authors' names
will be automatically generated.

\subsection{Correspondence note}

The footnote at the bottom of the first page is generated automatically,
and begins with the words `\emph{Correspondence and offprint
requests to}'.
The appropriate name and address must be given as an argument to the
\verb"\correspond" command in the preamble:
%
\begin{verbatim}
  \correspond{Christiane Notarmarco, Springer-Verlag London
              Limited, Sweetapple House, Catteshall Road,
              Godalming, Surrey GU7 3DJ, UK.
              e-mail: chris@svl.co.uk}
\end{verbatim}
%
This information is subsequently used by the \verb"\makecorrespond" command.
This may be placed anywhere after the \verb"\begin{document}" but within the
text of the first page. If you have other footnotes on the first page, the
\verb"\makecorrespond" command should follow them ensuring the output of this
command is printed at the foot of the page.

If you omit the \verb"\correspond" and \verb"\makecorrespond" commands,
the name in the correspondence note defaults to the short author's name.

\subsection{Abstract and keywords}

At the beginning of your article, the title should be generated in the
usual way using the \verb"\maketitle" command. Immediately following the
title you should include an abstract followed by a list of keywords. For
example, the titles for this guide were produced by the following source:
%
\begin{verbatim}
  \maketitle

  \begin{abstract}
  This guide is for authors who are preparing papers for the
  \emph{Formal Aspects of Computing} journal using the \LaTeXe\ %
  document-preparation system and the Formal Aspects of Computing
  class file (\texttt{fac.cls}).
  \end{abstract}

  \begin{keywords}
  \LaTeXe; Class file; \verb"fac.cls"; User guide
  \end{keywords}
\end{verbatim}

\subsection{Theorems}

The FAC class provides an extended \verb"\newtheorem" command which enables
you to generate unnumbered environments. e.g. An unnumbered `Theorem' can be typeset
with the following:
%
\begin{verbatim}
  \newtheorem{theorem}{Theorem}[section]
  :
  \begin{theorem}
  This gives a normal numbered theorem.
  \end{theorem}
  \begin{theorem*}
  This gives an unnumbered theorem.
  \end{theorem*}
\end{verbatim}
%
Which produces:
%
\begin{theorem}
This gives a normal numbered theorem.
\end{theorem}
\begin{theorem*}
This gives an unnumbered theorem.
\end{theorem*}
%
The preferred numbering scheme is for theorems to be numbered within sections,
as 1.1, 1.2, 1.3, etc. In order to allow
authors maximum flexibility in numbering and naming, \emph{no} theorem-like
environments are defined in \verb"fac.cls". Rather, you have to define
each one yourself. Theorem-like environments include Theorem, Definition,
Lemma, Corollary, and Proposition.

\subsection{Proofs}

A new environment exists for Proofs, e.g.
%
\begin{proof}
Noting that a state satisfying $VS$ has no elements
$V_{xi}$ of $V$ satisfying $0<V_{xi}<1$, it follows
that the proof is just as the proof was for the
corresponding case of \emph{Bi-States}.
\end{proof}
%
This was produced by the following code:
%
\begin{verbatim}
  \begin{proof}
  Noting that a state satisfying $VS$ has no elements
  $V_{xi}$ of $V$ satisfying $0<V_{xi}<1$, it follows
  that the proof is just as the proof was for the
  corresponding case of \emph{Bi-States}.
  \end{proof}
\end{verbatim}
%
The end of proof marker \proofbox\ is produced automatically. If you wish
to omit this, use the \verb"proof*" environment instead.

The proof environment can also take an optional argument which allows you
to produce `special' proofs, e.g.
%
  \begin{proof}[Proof of Theorem~1.1.]
  Noting that a state satisfying $VS$ has no elements
  $V_{xi}$ of $V$ satisfying $0<V_{xi}<1$, it follows
  that the proof is just as the proof was for the
  corresponding case of \emph{Bi-States}.
  \end{proof}
%
Which was produced like this:
%
\begin{verbatim}
  \begin{proof}[Proof of Theorem~1.1.]
  Noting that a state satisfying $VS$ has no elements
  $V_{xi}$ of $V$ satisfying $0<V_{xi}<1$, it follows
  that the proof is just as the proof was for the
  corresponding case of \emph{Bi-States}.
  \end{proof}
\end{verbatim}
%
Notice that once the optional argument is used, you have to type all of
the text which is to appear as the heading (including the full-stop).

\subsection{Lists}

The FAC class provides the three standard list environments:
\begin{itemize}
  \item Numbered lists, created using the \verb"enumerate" environment;
  \item Labelled lists, created using the \verb"description" environment;
  \item Bulleted lists, created using the \verb"itemize" environment.
\end{itemize}
The enumerated list numbers each list item with an arabic numeral.
Alternative styles can be achieved by inserting a redefinition of the
number labelling command after the \verb"\begin{enumerate}".
For example, a list numbered with roman numerals inside parentheses can be
produced by the following commands:
%
\begin{verbatim}
  \begin{enumerate}
    \renewcommand\theenumi{\roman{enumi}}
    \renewcommand\labelenumi{(\theenumi)}
    \item first item
          :
  \end{enumerate}
\end{verbatim}
%
This produces the following list:
%
\begin{enumerate}
  \renewcommand\theenumi{\roman{enumi}}
  \renewcommand\labelenumi{(\theenumi)}
  \item first item
  \item second item
  \item \etc\ldots
\end{enumerate}
%
In the last example, the labels were pushed out into the margin  because
the standard list indentation is designed to be sufficient for arabic
numerals rather than for the wider roman numerals. In order to enable
different labels to be used more easily, the \verb"enumerate" environment
in the FAC class can be given an optional argument which (like the standard
\verb"thebibliography" environment) specifies the \emph{widest label}. For
example,
%
\begin{enumerate}[(iii)]
  \renewcommand\theenumi{\roman{enumi}}
  \renewcommand\labelenumi{(\theenumi)}
  \item first item
  \item second item
  \item \etc\ldots
\end{enumerate}
%
was produced by the following input:
%
\begin{verbatim}
  \begin{enumerate}[(iii)]
    \renewcommand\theenumi{\roman{enumi}}
    \renewcommand\labelenumi{(\theenumi)}
    \item first item
          :
  \end{enumerate}
\end{verbatim}

\section{Mathematics}

The FAC class will set displayed mathematics with the correct indent
provided you use the \LaTeX\ standard of open and closed square brackets
as delimiters. The following equation
  \[ \sum_{i=1}^p \lambda_i = \mathrm{trace}(\mathbf{S}) \]
was typeset in the FAC style using the commands:
\begin{verbatim}
  \[ \sum_{i=1}^p \lambda_i = \mathrm{trace}(\mathbf{S}) \]
\end{verbatim}
Note the difference between the positioning of this equation and of
the following centred equation
  $$ \alpha_{j+1} > \bar{\alpha}+ks_{\alpha} $$
which was (wrongly) typeset using double dollars as follows:
\begin{verbatim}
  $$ \alpha_{j+1} > \bar{\alpha}+ks_{\alpha} $$
\end{verbatim}
Also, do not leave blank lines above and below displayed equations
unless a new paragraph is really intended.

\section{Typeface}

\subsection{AMS fonts -- especially if you do not have them}

If you need symbols from the AMS font set but do not have them installed,
you can ensure that they will be correctly typeset by taking the following
steps.
Set up user-defined macros that can be redefined by the typesetter to use
the correct AMS macros. For example, the blackboard bold symbols,
sometimes called shell or outline characters, are obtained with the AMS
macro \verb"\mathbb{..}". Instead, use a macro definition such as:
%
\begin{verbatim}
  % replace font!
  \newcommand\BbbE{\ensuremath{\mathsf{E}}} % Blackboard bold E
\end{verbatim}
%
This substitutes a sans serif character where you want blackboard
bold. You can typeset the input file and the typesetter is alerted to do the
substitution.

The following example (which uses the \verb"\providecommand" macro) will work
without modification by the typesetter, because the \verb"\providecommand"
macro will not overwrite any existing \verb"\mathbb" definition.
%
\begin{verbatim}
  \providecommand\mathbb[1]{\ensuremath{\mathsf{#1}}}
  ...
  \newcommand\BbbE{\mathbb{E}} % Blackboard bold E
\end{verbatim}

Plain \TeX\ provides only \verb"\leq" and \verb"\qeq" which typeset the
Computer Modern symbols $\realleq$ and $\realgeq$, respectively. These will
be redefined at typesetting to use the AMS symbols \verb"\leqslant" and
\verb"\geqslant", to give slanted symbols.

If you wish to use AMS fonts with \LaTeXe\ you must be using at least
version 2.0. Earlier versions are not supported.

\subsection{Sans serif symbols}

The \verb"\textsf" and \verb"\mathsf" commands change the typeface to
sans serif, giving upright characters. Occasionally, other sans serif
fonts are needed. You should use the following supplied macros to obtain
these other fonts.\\[6pt]
%
\verb"  \textsf{text}  " \lra \textsf{text}
  \qquad \verb"$\mathsf{math}$  " \lra $\mathsf{math}$\\
\verb"  \textsfi{text} " \lra \textsfi{text}
  \qquad \verb"$\mathsfi{math}$ " \lra $\mathsfi{math}$\\
\verb"  \textsfb{text} " \lra \textsfb{text}
  \qquad \verb"$\mathsfb{math}$ " \lra $\mathsfb{math}$\\
\verb"  \textsfbi{text}" \lra \textsfbi{text}
  \qquad \verb"$\mathsfbi{math}$" \lra $\mathsfbi{math}$\\[6pt]
%
You can use them like this:
%
\begin{verbatim}
  \newcommand\ssC{\mathsf{C}}     % for sans serif C
  \newcommand\sfsP{\mathsfi{P}}   % for sans serif slanted P
  \newcommand\sfbsX{\mathsfbi{X}} % for sans serif bold slanted X
\end{verbatim}
%
Note that the bold-slanted macros \verb"\textsfbi" and
\verb"\mathsfbi" use the slanted sans serif font \verb"cmssi"
-- because there is no bold-slanted math sans serif font in available in
Computer Modern!  If you use the supplied sans-serif text and math commands
the typesetter will be able to substitute the fonts automatically.

\subsection{Bold math italic symbols}

If you require bold math italic symbols/letters, \LaTeX\ provides several ways
of getting them.  Firstly, \LaTeX's \verb"\boldmath" switch which can be
used like this:
\begin{verbatim}
  $ \mbox{\boldmath $P$} = \mathsf{J}\mathcal{E} T $
\end{verbatim}
As you can see it takes quite alot of typing to achieve a bold math italic P.
Another problem is that you can't use \verb"\boldmath" in math mode, thus the
\verb"\mbox" forces the resulting text into text style (which you may not
want).

You can cut down on the typing by defining the following macro (which
is in the preamble of this guide):
\begin{verbatim}
  \providecommand\boldsymbol[1]{\mbox{\boldmath $#1$}}
\end{verbatim}
Then for the above example you can define:
\begin{verbatim}
  \newcommand\bldP{\boldsymbol{P}}
\end{verbatim}
the achieve the same result. This still doesn't remove the default to
text style problem.

If you have the \verb"amsbsy" package on your system, then you can remove
this limitation as well by placing a \verb"\usepackage{amsbsy}" after the
\verb"\documentclass" line in your document, and then use \verb"\boldsymbol"
for bold math italic symbols (don't define \verb"\boldsymbol" yourself when
using \verb"amsbsy").

\subsection{Script characters}

Script characters should be typeset using \LaTeXe's \verb"\mathcal"
command. This produces Computer Modern symbols such as $\mathcal{E}\,$
and $\mathcal{F}\,$ in your hard copy but the typesetter will substitute
the more florid script characters normally seen in the journal.

\subsection{Skewing of accents}

Accents such as hats, overbars and dots are normally centred over letters,
but when these are italic or sloping greek the accent may need to be moved
to the right so that it is centred over the top of the sloped letter. For
example,
%
\begin{verbatim}
  \newcommand\hatp{\skew3\hat{p}}
\end{verbatim}
will produce $\hatp$.

\section{User-defined macros}

If you define your own macros, you must ensure that their names do not
conflict with any existing macros in \LaTeX\ (or \AMSLaTeX\ %
if you are using this). You should also place them in the preamble of
your input file, between the \verb"\documentclass" line (but after any
\verb"\usepackage" commands) and before the \verb"\begin{document}" command.

Apart from scanning the indexes of the relevant manuals, you can check
whether a macro name is already used by using \verb"\newcommand", which
will check for the existence of the macro you are trying to define.
If the macro exists \LaTeX\ will respond with:
%
\begin{verbatim}
  ! LaTeX Error: Command ... already defined.
\end{verbatim}
%
In this case you should choose another name, and try again.

Such macros must be in a place where they can easily be found and
modified by the journal's editors or typesetter. They must be gathered
together in the preamble of your input file, or in a separate
\verb"macros.tex" file with the command \verb"\input{macros}" in the
preamble. Macro definitions must not be scattered about your document
where they are likely to be completely overlooked by the typesetter.

The same applies to font definitions that are based on Computer Modern
fonts. These must be changed by the typesetter to use the journal's
correct typeface. In this case, you should draw
attention to these font definitions on the hard copy that you submit for
publication and by placing a comment in your input file just before the
relevant definitions, for example \verb"% replace font!"

\section{Some guidelines for using standard facilities}

The following notes may help you achieve the best effects with the FAC
class file.

\subsection{Sections}

\LaTeX\ provides five levels of section headings and they are all
defined in the FAC class file:
%
\begin{itemize}
  \item[] Heading A -- \verb"\section{...}"
  \item[] Heading B -- \verb"\subsection{...}"
  \item[] Heading C -- \verb"\subsubsection{...}"
  \item[] Heading D -- \verb"\paragraph{...}"
  \item[] Heading E -- \verb"\subparagraph{...}"
\end{itemize}
%
Section numbers are given for sections, subsection and subsubsection
headings.

\subsection{Illustrations (or figures)}

The FAC class will cope with most positioning of your illustrations, and you
should not normally use the optional positional qualifiers on the
\verb"figure" environment which would override these decisions.
See `Instructions for authors' in the FAC journal for submission of artwork.
Figure captions should be below the figure itself therefore the \verb"\caption"
command should appear after the figure or space left for an illustration.
Any figure should be centred horizontally, this can be achived using the
\verb"\centering" command, or \verb"centerline" environment.
For example, Figure~\ref{sample-figure} is produced using the following.
%
\begin{verbatim}
  \begin{figure}
    \centering
    \vspace{5cm}
    \caption{An example figure in which space has been
             left for the artwork.}
    \label{sample-figure}
  \end{figure}
\end{verbatim}
%
\begin{figure}
  \centering
  \vspace{5cm}
  \caption{An example figure in which space has been
           left for the artwork.}
  \label{sample-figure}
\end{figure}

\subsection{Tables}

\begin{table}
  \caption{Index of Symbols}
  \begin{tabular}{@{}ll@{}}
   \hline
   Symbol\hspace{1cm}  & Meaning \\
   \hline
   \ifamsfontsloaded
     \whbl             & \hbox{Forward closure} \\
     \blwh             & \hbox{Backward closure} \\
     \blbl             & \hbox{Overlap closure} \\
     \whwh             & \hbox{Rule closure} \\
   \fi
   $\leadsto$          & \hbox{Rule closure constructor} \\
   $\bigtriangleup$    & \hbox{Substitution sum} \\
   $\cdot$             & \hbox{Substitution composition} \\
   $\sqcup$            & \hbox{Substitution join} \\
   $\bot$              & \hbox{Coherence relation} \\
   $-^n$               & \hbox{Substitution operator
                                         (exponent)} \\
   $W_n(-,-)$          & \hbox{Substitution operator
                                         (Whale)} \\
   $T_n(-,-,-)$        & \hbox{Substitution operator
                                         (Turtle)} \\
   \hline
  \end{tabular}
  \label{symbols}
\end{table}

The FAC class will cope with most positioning of your tables
and you should not normally use the optional positional qualifiers on the
\verb"table" environment which would override these decisions.
Table captions should be at the top, therefore the \verb"\caption" command
should appear before the body of the table.

The \verb"tabular" environment should be used to produce ruled tables;
it has been modified for the FAC class in the following ways.
%
\begin{enumerate}
  \item Rules may either be two-thirds or the full width of a page,
depending on the width of the material in your table. \verb"\hline"
will produce a rule two-thirds the width and \verb"\fullhline" will
produce a rule the full width of a page;
  \item Additional vertical space is inserted on either side of a rule;
  \item Vertical lines are not produced.
\end{enumerate}
%
Commands to redefine quantities such as \verb"\arraystretch" should be
omitted. For example, Table~\ref{symbols} is produced using the following.
%
\begin{verbatim}
  \begin{table}
    \caption{Index of Symbols}
    \begin{tabular}{@{}ll@{}}
     \hline
     Symbol\hspace{1cm}  & Meaning \\
     \hline
     \whbl               & \hbox{Forward closure} \\
     \blwh               & \hbox{Backward closure} \\
     \blbl               & \hbox{Overlap closure} \\
     \whwh               & \hbox{Rule closure} \\
     $\leadsto$          & \hbox{Rule closure constructor} \\
     $\bigtriangleup$    & \hbox{Substitution sum} \\
     $\cdot$             & \hbox{Substitution composition} \\
     $\sqcup$            & \hbox{Substitution join} \\
     $\bot$              & \hbox{Coherence relation} \\
     $-^n$               & \hbox{Substitution operator
                                           (exponent)} \\
     $W_n(-,-)$          & \hbox{Substitution operator
                                           (Whale)} \\
     $T_n(-,-,-)$        & \hbox{Substitution operator
                                           (Turtle)} \\
     \hline
    \end{tabular}
    \label{symbols}
  \end{table}
\end{verbatim}

\subsubsection{Continued captions}

If a table or figure will not fit onto a single page and has to broken into
more than one part, the subsequent parts should have a `continued' caption.
To achieve this, use \verb"\contcaption" instead of \verb"\caption", in the
subsequent figure or table environments. The \verb"\contcaption" uses exactly
the same syntax as the normal \verb"\caption" command, except it does not step
the figure or table counter.

\subsection{Appendices}

You should use the standard \LaTeX\ \verb"\appendix" command to place any
Appendices, normally, AFTER any acknowledgement and references.
From that point on \verb"\section" commands will produce appendixes, which are
numbered A, B etc. Equations, figures and tables continue in the same
numbering sequence as in the main text.

\subsection{References}

As with standard \LaTeX, there are two ways of producing a list of
references; either by compiling a list (using a \verb"thebibliography"
environment), or by using Bib\TeX\ with a suitable bibliographic database,
using the standard Bib\TeX\ `\verb"alpha"' style file.

\subsubsection{References in the text}

References to published literature should be quoted in text by an
abbreviation in square brackets of name(s) (three letters) and date
(two digits) -- this format is consistent with the Bib\TeX\ %
`\verb"alpha"' style.

Each entry has a key, which is assigned by the author
and used to refer to that entry in the text (with \verb"\cite").
e.g. \verb"\cite{Lam:LaTeX}".

A new form of the \verb"\cite" command is also provided `\verb"\nbcite"',
which works in the same way -- except it doesn't put any square brackets
in the output, allowing more complex citations formats to be constructed.

\subsubsection{The list of references}

The following listing shows some references prepared in the style of the
journal; the code produces the references at the end of this guide.
%
\begin{verbatim}
  \begin{thebibliography}{Lam94}
    \bibitem[But81]{But:Copy} Butcher, J.:
      \emph{Copy-editing: The Cambridge Handbook.}
      Cambridge University Press, 1981.

    \bibitem[Chi69]{Chicago}
      \emph{The Chicago Manual of Style.}
      University of Chicago Press, Chicago 60637, USA, 1982.

    \bibitem[For84]{For:Program} Forgaard, R.:
      A Program for Generating and Analyzing Term Rewriting
      Systems, Master's Thesis, MIT Lab. for Computer Science,
      1984.

    \bibitem[JLR82]{Jou:Recursive} Jouannaud, J. P., Lescanne, P.
      and Reinig, F.: Recursive Decomposition Ordering,
     \emph{Proc. Conf. on Formal Description of Programming
      Concepts II}, pp. 331--346, 1982.

    \bibitem[Lam94]{Lam:LaTeX} Lamport, L.:
      \emph{\LaTeX: A Document Preparation System (2nd edition).}
      Addison-Wesley, New York, 1994.

    \bibitem[Ped85]{Ped:Obtaining} Pederson, J.:
      Obtaining Complete Sets of Reproductions and Equations
      without Using Special Unification Algorithms. Unpublished
      manuscript, 1985.

    \bibitem[PeS81]{Pet:Complete} Peterson, G. E. and Stickel,
      M. E.: Complete Sets of Reductions for Some Equational
      Theories. \emph{J. ACM}, \textbf{28}, 223--264 (1981).
  \end{thebibliography}
\end{verbatim}
%
Where more than one reference is cited having the author(s) and date, the
letters a,b,c, \ldots\ should follow the date (\eg [Smi88a], [Smi88b],
\etc.). References should be listed in \verb"thebibliography" environment
alphabetically by author(s)' name(s) and then by year if the same author
has several papers.

Formatting for italic etc.\ should be avoided unless you are sure you
understand the style of references; please concentrate on giving full and
clear information.

\begin{thebibliography}{Lam94}
  \bibitem[But81]{But:Copy} Butcher, J.:
    \emph{Copy-editing: The Cambridge Handbook.}
    Cambridge University Press, 1981.

  \bibitem[Chi69]{Chicago}
    \emph{The Chicago Manual of Style.}
    University of Chicago Press, Chicago 60637, USA, 1982.

  \bibitem[For84]{For:Program} Forgaard, R.:
    A Program for Generating and Analyzing Term Rewriting
    Systems, Master's Thesis, MIT Lab. for Computer Science,
    1984.

  \bibitem[JLR82]{Jou:Recursive} Jouannaud, J. P., Lescanne, P.
    and Reinig, F.: Recursive Decomposition Ordering,
   \emph{Proc. Conf. on Formal Description of Programming
    Concepts II}, pp. 331--346, 1982.

  \bibitem[Lam94]{Lam:LaTeX} Lamport, L.:
    \emph{\LaTeX: A Document Preparation System (2nd edition).}
    Addison-Wesley, New York, 1994.

  \bibitem[Ped85]{Ped:Obtaining} Pederson, J.:
    Obtaining Complete Sets of Reproductions and Equations
    without Using Special Unification Algorithms. Unpublished
    manuscript, 1985.

  \bibitem[PeS81]{Pet:Complete} Peterson, G. E. and Stickel,
    M. E.: Complete Sets of Reductions for Some Equational
    Theories. \emph{J. ACM}, \textbf{28}, 223--264 (1981).
\end{thebibliography}

\newpage

\appendix
\section{Special commands in \texttt{fac.cls}}

The following is a summary of the new commands, optional arguments and
environments which have been added to the standard \LaTeX\ user-interface
in creating the FAC class file.
\vspace{6pt}

\noindent
\begin{tabular}{@{}p{38mm}@{}p{127mm}@{}}
\emph{New commands}    & \\[3pt]
\ifprodtf
\verb"\acceptedline"   & sets the `Accepted in revised form' text.\\
\fi
\verb"\contcaption"    & produces a continued figure or table caption. e.g.\\
                       & `Figure 1---{\itshape cont.}' This command is used in the
                         same way as the standard \verb"\caption" command.\\
\verb"\correspond"     & sets the text used for the `Correspondence and offprint
                         requests to' footnote on the first page.\\
\ifprodtf
\verb"\editor"         & sets the editors name for a `Short Communication' paper.\\
\fi
\verb"\makecorrespond" & ensures the `Correspondence and offprint requests to'
                         footnote is typeset. This command is usually used just
                         before \verb"\maketitle".\\
\verb"\mathproofbox"   & as \verb"\proofbox", except this version is intended
                         for use in equations ending proof's (it typesets the
                         proof box using \verb"\rlap" with 1em of extra space).\\
\verb"\nbcite"         & works in the same way as the normal \verb"\cite"
                         command except it doesn't put in the `[ ]'s.\\
\verb"\newtheorem"     & has been modified to create both numbered and unnumbered
                         `theorem' environments.\\
\verb"\proofbox"       & typesets a proof box \proofbox\ (this is normally
                         put in automatically at the end of the \verb"proof"
                         environment). If you need to insert a \verb"\proofbox"
                         manually, you should add a `\verb"\quad"' of
                         space before it in the output.\\
\ifprodtf
\verb"\pagerange"      & sets the range of pages used by the paper.\\
\verb"\pubyear"        & sets the publication year.\\
\verb"\receivedline"   & sets the `Received' date information.\\
\fi
\verb"\removebrackets" & removes the `(\ )' brackets from the optional
                         argument of environments created by the
                         \verb"\newtheorem" command. Should be placed
                         just before the appropriate environment.\\
\ifprodtf
\verb"\removefullpoint" & removes the full point from the next \verb"\caption"
                          command, usually used for continued captions.\\
\verb"\shortcom"       & specifies that this is to be a `Short Communication' paper.\\
\verb"\volume"         & sets the volume number.\\[6pt]
\else
  \noalign{\vskip 6pt}
\fi
\emph{New environments} & \\[3pt]
\ifprodtf
\verb"bottomfigure"     & for split figures and captions (on facing page).\\
\fi
\verb"proof" & to typeset mathematical proofs, the \verb"*"-form omits
               the proof box.\\
\verb"tabular"          & has been modified to insert additional space above
                          and below an \verb"\hline", the table caption
                          and body is centred with rules full out across the
                          text measure. The \verb"tabular" environments old
                          behaviour can be restored by using the
                          \verb"oldtabular" environment.\\[6pt]
\emph{New optional arguments} & \\[3pt]
\verb"[<short author>]" & in the \verb"\author" command: to define a left
                          running headline that is different from the
                          authors' names as typeset at the article opening.
                          \ifprodtf The \verb"\shortauthor" command is
                          also provided.\fi \\
\verb"[<short title>]"  & in the \verb"\title" command: to define a right
                          running headline that is different from the article
                          title. \ifprodtf The \verb"\shorttitle" command is
                          also provided.\fi \\
\verb"[<widest label>]" & in \verb"\begin{enumerate}": to ensure the correct
                          alignment of numbered lists.\\
\end{tabular}

\ifprodtf
\section{Notes for editors}

This appendix contains additional information which may be useful to
those who are involved with the final production stages of an article.
Authors, who are generally not typesetting the final pages in the
journal's typeface (Monotype Times), do not need this information.

\subsection{Setting the production typeface}

The global \verb"\documentclass" option `\verb"prodtf"' sets up
\verb"fac.cls" to typeset in the production typeface -- Monotype Times.
e.g.
%
\begin{verbatim}
  \documentclass[prodtf]{fac}
\end{verbatim}

\subsection{Catchline commands}

To be placed in the preamble:
\begin{itemize}
\item \verb"\pubyear{}"
\item \verb"\pagerange{}"
\item \verb"\volume{}"
\end{itemize}

\subsection{Footnotes}

If a footnote falls at the bottom of a page, it is possible for the
footnote to appear on the following page (a feature of \TeX ). Check
for this.

\subsection{Landscape material}

The add on package \verb"FACland" provides macros for landscape figures and
tables. See the \verb"FACland" guide for further information.

\subsection{Figures with split artwork/captions}

When a figure is too large to fit on a page with it's caption, you can use
the following procedure to place the figure, and then it's caption at the foot
of the facing page. First set the figure with a short caption
using the normal \verb"figure" environment, e.g.
\begin{verbatim}
  \begin{figure}
    ...
    \caption{For caption see facing page.}
  \end{figure}
\end{verbatim}
Then set the correct (long) caption, so that it appears on the facing page:
\begin{verbatim}
  \begin{bottomfigure}
    \addtocounter{figure}{-1}
    \caption{This is the long caption...}
  \end{bottomfigure}
\end{verbatim}
If the figure falls on a recto, you may have to move the \verb"bottomfigure"
environment to before the \verb"figure" environment. In this case you need to
move the \verb"\addtocounter" command into the \verb"figure" environment
instead.

The \verb"bottomfigure" environment places a full measure rule above the
bottom-caption automatically.

\section{Macros provided by \texttt{FAC2esym.sty}}

\subsection{Automatic font/character changes}

\begin{itemize}\itemsep=6pt
\item The \verb|\le|, \verb|\leq|, \verb|\ge|, \verb|\geq| commands
use the equivalent AMS slanted symbols:
\[
\oldle \oldleq \oldge \oldgeq
 \lra
\le \leq \ge \geq
\]
The normal characters can be obtained by using the \verb|\old| form
(\eg \verb|\oldge|).
\end{itemize}

\subsection{Additional fonts}

\begin{itemize}\itemsep=6pt
\item The complete AMS symbols are available using the normal names:
\[
\hbox{\verb"\boxdot \boxplus \boxtimes"} \lra \boxdot \boxplus \boxtimes
\]

\item Blackboard bold:
\[
\hbox{\verb"$\mathbb{ABC}$"} %%% \lra \mathbb{ABC}
\]

\item Fraktur/Gothic (bold math version available):
\[
   \hbox{\verb"$\mathfrak{ABC}$"} %%% \lra \mathfrak{ABC}
\]

\item Monotype Script (bold math version available):
%
\[ \verb"$\mathscr{ABCabc}$ " \lra \mathscr{ABCabc}  \]
\[ \verb"$\mathbscr{ABCabc}$" \lra \mathbscr{ABCabc} \]
You may still use \verb"\mathcal" to obtain upper-case Script characters.

\item Upright Greek: The \verb"\mathup" and \verb"\mathbup" macros are provided to obtain
upright lowercase Greek characters.
%
\[ \verb"$\mathup{\alpha}$ " \lra \mathup{\alpha}  \]
\[ \verb"$\mathbup{\alpha}$" \lra \mathbup{\alpha} \]
%
All of the lower-case Greek letters are pre-defined.

\item Bold math italic/symbols are provided by the \verb"\boldsymbol" macro
(from the \verb"amsbsy" package). The \verb"\bmath" macro is provided as
an alias.

The \verb"FAC2esym" package also defines most of the
symbols from Appendix F of the \TeX book. These can be obtained by using
their normal (unbold) symbol name prefixed with a `b'. e.g.\ \verb|\nabla|
becomes \verb|\bnabla|. The only exception to this rule is \verb|\eta|,
which whould lead to a clash with \verb|\beta|. In this case use
\verb|\boldeta| for bold eta.

The problem with space disappearing around certain
bold math symbols (\verb"\bcdot") does not happen, when using the
\verb"amsbsy" package.

\item The \verb"\textsf*" and \verb"\mathsf*" macros are redefined to
use the correct Monotype Helvetica fonts.
\end{itemize}
\fi

\label{lastpage}

\end{document}
